% Options for packages loaded elsewhere
\PassOptionsToPackage{unicode}{hyperref}
\PassOptionsToPackage{hyphens}{url}
%
\documentclass[
]{article}
\usepackage{amsmath,amssymb}
\usepackage{iftex}
\ifPDFTeX
  \usepackage[T1]{fontenc}
  \usepackage[utf8]{inputenc}
  \usepackage{textcomp} % provide euro and other symbols
\else % if luatex or xetex
  \usepackage{unicode-math} % this also loads fontspec
  \defaultfontfeatures{Scale=MatchLowercase}
  \defaultfontfeatures[\rmfamily]{Ligatures=TeX,Scale=1}
\fi
\usepackage{lmodern}
\ifPDFTeX\else
  % xetex/luatex font selection
\fi
% Use upquote if available, for straight quotes in verbatim environments
\IfFileExists{upquote.sty}{\usepackage{upquote}}{}
\IfFileExists{microtype.sty}{% use microtype if available
  \usepackage[]{microtype}
  \UseMicrotypeSet[protrusion]{basicmath} % disable protrusion for tt fonts
}{}
\makeatletter
\@ifundefined{KOMAClassName}{% if non-KOMA class
  \IfFileExists{parskip.sty}{%
    \usepackage{parskip}
  }{% else
    \setlength{\parindent}{0pt}
    \setlength{\parskip}{6pt plus 2pt minus 1pt}}
}{% if KOMA class
  \KOMAoptions{parskip=half}}
\makeatother
\usepackage{xcolor}
\usepackage[margin=1in]{geometry}
\usepackage{color}
\usepackage{fancyvrb}
\newcommand{\VerbBar}{|}
\newcommand{\VERB}{\Verb[commandchars=\\\{\}]}
\DefineVerbatimEnvironment{Highlighting}{Verbatim}{commandchars=\\\{\}}
% Add ',fontsize=\small' for more characters per line
\usepackage{framed}
\definecolor{shadecolor}{RGB}{248,248,248}
\newenvironment{Shaded}{\begin{snugshade}}{\end{snugshade}}
\newcommand{\AlertTok}[1]{\textcolor[rgb]{0.94,0.16,0.16}{#1}}
\newcommand{\AnnotationTok}[1]{\textcolor[rgb]{0.56,0.35,0.01}{\textbf{\textit{#1}}}}
\newcommand{\AttributeTok}[1]{\textcolor[rgb]{0.13,0.29,0.53}{#1}}
\newcommand{\BaseNTok}[1]{\textcolor[rgb]{0.00,0.00,0.81}{#1}}
\newcommand{\BuiltInTok}[1]{#1}
\newcommand{\CharTok}[1]{\textcolor[rgb]{0.31,0.60,0.02}{#1}}
\newcommand{\CommentTok}[1]{\textcolor[rgb]{0.56,0.35,0.01}{\textit{#1}}}
\newcommand{\CommentVarTok}[1]{\textcolor[rgb]{0.56,0.35,0.01}{\textbf{\textit{#1}}}}
\newcommand{\ConstantTok}[1]{\textcolor[rgb]{0.56,0.35,0.01}{#1}}
\newcommand{\ControlFlowTok}[1]{\textcolor[rgb]{0.13,0.29,0.53}{\textbf{#1}}}
\newcommand{\DataTypeTok}[1]{\textcolor[rgb]{0.13,0.29,0.53}{#1}}
\newcommand{\DecValTok}[1]{\textcolor[rgb]{0.00,0.00,0.81}{#1}}
\newcommand{\DocumentationTok}[1]{\textcolor[rgb]{0.56,0.35,0.01}{\textbf{\textit{#1}}}}
\newcommand{\ErrorTok}[1]{\textcolor[rgb]{0.64,0.00,0.00}{\textbf{#1}}}
\newcommand{\ExtensionTok}[1]{#1}
\newcommand{\FloatTok}[1]{\textcolor[rgb]{0.00,0.00,0.81}{#1}}
\newcommand{\FunctionTok}[1]{\textcolor[rgb]{0.13,0.29,0.53}{\textbf{#1}}}
\newcommand{\ImportTok}[1]{#1}
\newcommand{\InformationTok}[1]{\textcolor[rgb]{0.56,0.35,0.01}{\textbf{\textit{#1}}}}
\newcommand{\KeywordTok}[1]{\textcolor[rgb]{0.13,0.29,0.53}{\textbf{#1}}}
\newcommand{\NormalTok}[1]{#1}
\newcommand{\OperatorTok}[1]{\textcolor[rgb]{0.81,0.36,0.00}{\textbf{#1}}}
\newcommand{\OtherTok}[1]{\textcolor[rgb]{0.56,0.35,0.01}{#1}}
\newcommand{\PreprocessorTok}[1]{\textcolor[rgb]{0.56,0.35,0.01}{\textit{#1}}}
\newcommand{\RegionMarkerTok}[1]{#1}
\newcommand{\SpecialCharTok}[1]{\textcolor[rgb]{0.81,0.36,0.00}{\textbf{#1}}}
\newcommand{\SpecialStringTok}[1]{\textcolor[rgb]{0.31,0.60,0.02}{#1}}
\newcommand{\StringTok}[1]{\textcolor[rgb]{0.31,0.60,0.02}{#1}}
\newcommand{\VariableTok}[1]{\textcolor[rgb]{0.00,0.00,0.00}{#1}}
\newcommand{\VerbatimStringTok}[1]{\textcolor[rgb]{0.31,0.60,0.02}{#1}}
\newcommand{\WarningTok}[1]{\textcolor[rgb]{0.56,0.35,0.01}{\textbf{\textit{#1}}}}
\usepackage{graphicx}
\makeatletter
\def\maxwidth{\ifdim\Gin@nat@width>\linewidth\linewidth\else\Gin@nat@width\fi}
\def\maxheight{\ifdim\Gin@nat@height>\textheight\textheight\else\Gin@nat@height\fi}
\makeatother
% Scale images if necessary, so that they will not overflow the page
% margins by default, and it is still possible to overwrite the defaults
% using explicit options in \includegraphics[width, height, ...]{}
\setkeys{Gin}{width=\maxwidth,height=\maxheight,keepaspectratio}
% Set default figure placement to htbp
\makeatletter
\def\fps@figure{htbp}
\makeatother
\setlength{\emergencystretch}{3em} % prevent overfull lines
\providecommand{\tightlist}{%
  \setlength{\itemsep}{0pt}\setlength{\parskip}{0pt}}
\setcounter{secnumdepth}{-\maxdimen} % remove section numbering
\ifLuaTeX
  \usepackage{selnolig}  % disable illegal ligatures
\fi
\IfFileExists{bookmark.sty}{\usepackage{bookmark}}{\usepackage{hyperref}}
\IfFileExists{xurl.sty}{\usepackage{xurl}}{} % add URL line breaks if available
\urlstyle{same}
\hypersetup{
  hidelinks,
  pdfcreator={LaTeX via pandoc}}

\author{}
\date{\vspace{-2.5em}}

\begin{document}

title: ``RWorksheet\_Deluna\#3B'' author: ``Nicole De Luna'' date:
``2023-10-19'' output: pdf\_document

\hypertarget{r-markdown}{%
\subsection{R Markdown}\label{r-markdown}}

This is an R Markdown document. Markdown is a simple formatting syntax
for authoring HTML, PDF, and MS Word documents. For more details on
using R Markdown see \url{http://rmarkdown.rstudio.com}.

When you click the \textbf{Knit} button a document will be generated
that includes both content as well as the output of any embedded R code
chunks within the document. You can embed an R code chunk like this:

\begin{Shaded}
\begin{Highlighting}[]
\FunctionTok{summary}\NormalTok{(cars)}
\end{Highlighting}
\end{Shaded}

\begin{verbatim}
##      speed           dist       
##  Min.   : 4.0   Min.   :  2.00  
##  1st Qu.:12.0   1st Qu.: 26.00  
##  Median :15.0   Median : 36.00  
##  Mean   :15.4   Mean   : 42.98  
##  3rd Qu.:19.0   3rd Qu.: 56.00  
##  Max.   :25.0   Max.   :120.00
\end{verbatim}

\hypertarget{including-plots}{%
\subsection{Including Plots}\label{including-plots}}

You can also embed plots, for example:

\includegraphics{RWorksheet_Deluna-3B_files/figure-latex/pressure-1.pdf}

Note that the \texttt{echo\ =\ FALSE} parameter was added to the code
chunk to prevent printing of the R code that generated the plot.

\#1. Create a data frame using the table below

\#1A. Write the codes

HouseholdData \textless- data.frame( Respond\_1 = c(1:20),

Sex =
c(``Female'',``Female'',``Male'',``Female'',``Female'',``Female'',``Female'',``Female'',``Female'',``Female'',``Male'',``Female'',``Fe
male'',``Female'',``Female'',``Female'',``Female'',``Female'',``Male'',``Female''),

FatherOccupation =
c(``Farmer'',``Others'',``Others'',``Others'',``Farmer'',``Driver'',``Others'',``Farmer'',``Farmer'',``Farmer'',``Oth
ers'',``Driver'',``Farmer'',``Others'',``Others'',``Farmer'',``Others'',``Farmer'',``Driver'',``Farmer''),

PersonAtHome = c(5,7,3,8,5,9,6,7,8,4,7,5,4,7,8,8,3,11,7,6),

SiblingsAtSchool = c(6,4,4,1,2,1,5,3,1,2,3,2,5,5,2,1,2,5,3,2),

TypesOfHouses =
c(``Wood'',``Semi-Concrete'',``Concrete'',``Wood'',``Wood'',``Concrete'',``Concrete'',``Wood'',``Semi-Concrete'',``Con
crete'',``Semi-Concrete'',``Concrete'',``Semi-Concrete'',``Semi-Concrete'',``Concrete'',``Concrete'',``Concrete'',``Concrete'',``Conc
rete'',``Semi-Concrete'')

) HouseholdData

\#1b. Describe the data. Get the structure or the summary of the data

summary(HouseholdData)

\#1c Is the mean number of siblings attending is 5? MeanNumSib
\textless- mean(HouseholdData\$SiblingsAtSchool) IsMean5 \textless-
MeanNumSib == 5 print(IsMean5)

\hypertarget{false}{%
\section{False}\label{false}}

\#1d. Extract the 1st two rows and then all the columns using the
subsetting functions.Write the codes and its output.

ftrac \textless- HouseholdData{[}1:2, {]} print(ftrac)

\#1e. Extract 3rd and 5th row with 2nd and 4th column. Write the codes
and its result.

selrowcol \textless- HouseholdData{[}c(3, 5), c(2, 4){]}
print(selrowcol)

\#1f. Select the variable types of houses then store the vector that
results as types\_houses.Write the codes.

types\_houses \textless- HouseholdData\$TypeOfHouses

\#1g. Select only all Males respondent that their father occupation was
farmer. Write thecodes and its output.

MaleFarmers \textless-
HouseholdData{[}HouseholdData\(Sex =="Male" & HouseholdData\)FatherOccupation
== ``Farmer'', {]} print(MaleFarmers)

\#1h. Select only all females respondent that have greater than or equal
to 5 number ofsiblings attending school. Write the codes and its
outputs.

FemGT5Sib \textless-
HouseholdData{[}HouseholdData\(Sex == "Female" & HouseholdData\)SiblingsAtSchool
\textgreater= 5, {]} print(FemGT5Sib)

\#2 Write a R program to create an empty data frame. Using the following
codes:

df = data.frame(Ints=integer(), Doubles=double(),
Characters=character(), Logicals=logical(), Factors=factor(),
stringsAsFactors=FALSE)

print(``Structure of the empty dataframe:'') print(str(df))

\#2a. The data frame is empty

\#3 HouseholdData \textless- data.frame( Respondents = c(1:10), Sex =
c(``Male'', ``Female'', ``Female'', ``Male'', ``Male'', ``Female'',
``Female'', ``Male'', ``Female'', ``Male''), FathersOccupation = c(1, 2,
3, 3, 1, 2, 2, 3, 1, 3), PersonAtHome = c(5, 7, 3, 8, 6, 4, 4, 2, 11,
6), SiblingsAtSchool = c(2, 3, 0, 5, 2, 3, 1, 2, 6, 2), TypesOfHouse =
c(``Wood'', ``Congrete'', ``Congrete'', ``Wood'', ``Semi-Congrete'',
``Semi-Congrete'', ``Wood'', ``Semi-Congrete'', ``Semi-Congrete'',
``Congrete'') ) HouseholdData

\#3a.Import the csv file into the R environment. Write the codes.

write.csv(HouseholdData, file = ``HouseholdData.csv'', row.names =
FALSE) ImporteData \textless- read.csv(``HouseholdData.csv'')

\#3b.. Convert the Sex into factor using factor() function and change it
into integer.{[}Legend:Male = 1 and Female = 2{]}. Write the R codes and
its output.

ImporteData\(Sex <- factor(ImporteData\)Sex, levels = c(``Male'',
``Female'')) ImporteData\(Sex <- as.integer(ImporteData\)Sex)

\#3c.Convert the Type of Houses into factor and change it into integer.
{[}Legend: Wood= 1; Congrete = 2; Semi-Congrete = 3{]}. Write the R
codes and its output.
ImporteData\(TypesOfHouse <- factor(ImporteData\)TypesOfHouse, levels =
c(``Wood'', ``Concrete'', ``Semi-Concrete''))
ImporteData\(TypesOfHouse <- as.integer(ImporteData\)TypesOfHouse)

ImporteData\(TypesOfHouse <- factor(ImporteData\)TypesOfHouse, levels =
c(``Wood'', ``Concrete'', ``Semi-Concrete''))
ImporteData\(TypesOfHouse <- as.integer(ImporteData\)TypesOfHouse)

\#3d.On father's occupation, factor it as Farmer = 1; Driver = 2; and
Others = 3. Whatis the R code and its output?

ImporteData\(FathersOccupation <- factor(ImporteData\)FathersOccupation,
levels = c(``Farmer'', ``Driver'', ``Others''))
ImporteData\(FathersOccupation <- as.integer(ImporteData\)FathersOccupation)

\#3e.Select only all females respondent that has a father whose
occupation is driver. Writethe codes and its output. FemaleDrivers
\textless-
ImporteData{[}ImporteData\(Sex == 2 & ImporteData\)FathersOccupation ==
2, {]} print(FemaleDrivers)

\#3f.Select the respondents that have greater than or equal to 5 number
of siblings attendingschool. Write the codes and its output. GT5Sib
\textless- ImporteData{[}ImporteData\$SiblingsAtSchool \textgreater= 5,
{]} print(GT5Sib)

\#4. Interpret the graph

\hypertarget{the-graph-in-figure-3-represents-the-sentiments-of-people-every-day-that-has-a-major-impact-on-our-world-everytime.-in-short-we-always-allow-ourselves-to-show-our-opinions-and-our-knowledge-that-we-learn-in-a-all-day-basis.}{%
\section{The graph in figure 3 represents the sentiments of people every
day that has a major impact on our world everytime. In short, we always
allow ourselves to show our opinions and our knowledge that we learn in
a all day
basis.}\label{the-graph-in-figure-3-represents-the-sentiments-of-people-every-day-that-has-a-major-impact-on-our-world-everytime.-in-short-we-always-allow-ourselves-to-show-our-opinions-and-our-knowledge-that-we-learn-in-a-all-day-basis.}}

\end{document}
